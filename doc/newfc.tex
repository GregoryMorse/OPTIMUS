%% LyX 2.3.6 created this file.  For more info, see http://www.lyx.org/.
%% Do not edit unless you really know what you are doing.
\documentclass[english]{article}
\usepackage[T1]{fontenc}
\usepackage[latin9]{inputenc}
\usepackage[active]{srcltx}
\usepackage{float}
\usepackage{url}
\usepackage{amsmath}

\makeatletter

%%%%%%%%%%%%%%%%%%%%%%%%%%%%%% LyX specific LaTeX commands.
%% Because html converters don't know tabularnewline
\providecommand{\tabularnewline}{\\}
\floatstyle{ruled}
\newfloat{algorithm}{tbp}{loa}
\providecommand{\algorithmname}{Algorithm}
\floatname{algorithm}{\protect\algorithmname}

%%%%%%%%%%%%%%%%%%%%%%%%%%%%%% Textclass specific LaTeX commands.
\newenvironment{lyxcode}
	{\par\begin{list}{}{
		\setlength{\rightmargin}{\leftmargin}
		\setlength{\listparindent}{0pt}% needed for AMS classes
		\raggedright
		\setlength{\itemsep}{0pt}
		\setlength{\parsep}{0pt}
		\normalfont\ttfamily}%
	 \item[]}
	{\end{list}}

\makeatother

\usepackage{babel}
\begin{document}
\title{Constructing features using a hybrid genetic algorithm}
\author{Ioannis G. Tsoulos\thanks{Corresponding author. Email: itsoulos@uoi.gr}}
\date{Department of Informatics and Telecommunications, University of Ioannina,
Greece}
\maketitle
\begin{abstract}
A hybrid procedure that incorporates Grammatical Evolution and a weight
decaying technique is proposed here for classification and regression
problems. The procedure is divided in two main phases: in the first
phase new features are created from the original ones and in second
phase a hybrid genetic algorithm is used to train neural networks
on the previously created features. The proposed procedure is applied
on a wide range of classification and regression problems from the
relevant literature and the results are reported and discussed.
\end{abstract}
\textbf{Keywords}: Genetic algorithm, machine learning, neural networks,
Grammatical Evolution.

\section{Introduction}

Neural networks are well - established parametric tools \cite{nn1,nn2},
used with success in many areas such as physics \cite{nnphysics1,nnphysics2,nnphysics3},
chemistry \cite{nnchem1,nnchem2,nnchem3}, economics \cite{nnecon1,nnecon2,nncecon3},
medicine \cite{nnmed1,nnmed2} etc. A neural network usually is expressed
as a function $N(\overrightarrow{x},\overrightarrow{w})$, where$\overrightarrow{x}$
is the input vector and $\overrightarrow{w}$ is the weight vector
to be calculated through some optimization process. This optimization
process should estimate the weight vector by minimizing the following
error function:
\begin{equation}
E\left(N\left(\overrightarrow{x},\overrightarrow{w}\right)\right)=\sum_{i=1}^{M}\left(N\left(\overrightarrow{x}_{i},\overrightarrow{w}\right)-y_{i}\right)^{2}\label{eq:eq1}
\end{equation}
In the above equation the set $\left(\overrightarrow{x_{i}},y_{i}\right),\ i=1,...,M$
is the data used to train the neural network, where symbol $y_{i}$
represents the actual output for the point $\overrightarrow{x_{i}}$
. In many cases the neural network $N(\overrightarrow{x},\overrightarrow{w})$
can be expressed as a weighted summation of processing units as proposed
in \cite{nnc} and defined such as:
\begin{equation}
N\left(\overrightarrow{x},\overrightarrow{w}\right)=\sum_{i=1}^{H}w_{(d+2)i-(d+1)}\sigma\left(\sum_{j=1}^{d}x_{j}w_{(d+2)i-(d+1)+j}+w_{(d+2)i}\right)\label{eq:nn}
\end{equation}
where $H$ is the number of processing units of the neural network
and $d$ is the dimension of vector $\overrightarrow{x}$ and $\sigma(x)$
is the sigmoid function defined as: 
\begin{equation}
\sigma(x)=\frac{1}{1+\exp(-x)}\label{eq:sig}
\end{equation}
From the equation \ref{eq:nn} one can obtain that the dimension of
weight vector $w$ is computed as: $w=(d+2)H$. The function of equation
\ref{eq:eq1} has been minimized with a variety of optimization methods
during the past years such as: the Back Propagation method \cite{bpnn,bpnn2},
the RPROP method \cite{rpropnn,rpropnn3,rpropnn2}, Quasi Newton methods
\cite{quasinn,quasinn2}, Genetic Algorithms \cite{geneticnn,geneticnn2},
Particle Swarm Optimization \cite{psonn,psonn2} etc. All the previously
mentioned methods have to overcome two major problems:
\begin{itemize}
\item Excessive computational times, because they require processing time
proportional to the dimension of the objective problem and the number
of processing units as well. For example, a neural network of $H=10$
processing units applied to a test data with $d=3$, is considered
as an optimization problem with dimension $w=(d+2)H=50$. A discussion
on the effects of the dimensionality on neural networks is provided
in \cite{nndimension}. A common approach to overcome this problem
is to use the PCA technique to reduce the dimensionality of the objective
problem \cite{nnpca1,nnpca2,nnpca3} i.e. the parameter $d$.
\item The ovetfitting problem. It is common for these methods to produce
poorer results when they are applied to data (test data) not previously
used in the optimization process. This problem is discussed in detail
in the article of Geman et all \cite{nngeman} as well as in the article
of Hawkins \cite{nnhawkins}. A variety of methods have been proposed
to overcome this problem such as are weight sharing \cite{nnsharing1},
pruning \cite{nnprunning1,nnprunning2,nnprunning3}, the dropout technique
\cite{nndrop1}, early stopping \cite{nnearly1,nnearly2}, and weight
decaying \cite{nndecay1,nndecay2}. 
\end{itemize}
This article proposes a method that tackle both the above problems
using two major steps. During the first step new features are created
from the original ones using a procedure based on the Grammatical
Evolution technique \cite{ge1}. This procedure was introduced in
the work of Gavrilis et al \cite{fc1} and it has been with success
in many areas such as Spam Identification\textbf{ }\cite{fc2}\textbf{,
}Fetal heart classification\textbf{ }\cite{fc3}\textbf{, }epileptic
oscillations in clinical intracranial electroencephalograms \cite{fc4}
etc. The outcomes of the first phase are the modified training and
testing data according to the created features. During the second
step, a genetic algorithm that incorporates a weight decaying procedure
is used to train a neural network on the modified data of the first
step.

The rest of this article is organized as follows: in section \ref{sec:Method-description}
the proposed method is described in detail, in section \ref{sec:Experiments}
the proposed method is tested on a series of well know datasets from
the relevant literature and the results are compared to those of a
simple genetic algorithm and finally in section \ref{sec:Conclusions}
some conclusions are presented.

\section{Method description\label{sec:Method-description}}

The proposed method has two major phases. During the first phase a
procedure based on the Grammatical Evolution technique is utilized
in order to create new features from the old ones. The new features
are evaluated using an RBF \cite{rbf1} neural network with $H$ hidden
nodes. The RBF network is used during this phase instead of a neural
network because the training procedure for RBF networks are must faster
than these of neural networks of the Equation \ref{eq:nn}. In the
second phase, a hybrid genetic algorithm is to a neural network trained
using the constructed features of the first phase. 

\subsection{The usage of Grammatical Evolution\label{subsec:Transformation-procedure}}

Grammatical evolution is a genetic programming procedure where the
chromosomes are represent production rules from a BNF grammar. The
production procedure starts from the start symbol of the grammar and
produces programs by replacing non terminal symbols with the right
hand of the production rules that will be selected according to the
value of each element in the chromosome. In the proposed method the
BNF grammar of the Figure \ref{fig:BNF-grammar-of} was used to create
a new feature from the original ones. The parameter N denotes the
number of original features. $[0,255].$ For example, consider the
chromosome $x=\left[9,8,6,4,16,10,17,23,8,14\right]$ and $N=3$.
The steps to produce the valid expression $f(x)=x_{2}+\cos\left(x_{3}\right)$
are listed in Table \ref{tab:table_with_steps}.\textbf{ }Each number
in the parentheses denotes the sequence number of the production rule.
The process to produce $N_{f}$ features from the original  have as
follows:
\begin{enumerate}
\item Every chromosome $Z$ is split into $N_{f}$ parts. Each part $g_{i}$
will be used to construct a feature.
\item For every part $g_{i}$ construct a feature $t_{i}$ using the grammar
given in \ref{fig:BNF-grammar-of}
\item Create a mapping function 
\begin{equation}
G(\overrightarrow{x},Z)=\left(t_{1}\left(\overrightarrow{x},Z\right),t_{2}\left(\overrightarrow{x},Z\right),\ldots,t_{N_{f}}\left(\overrightarrow{x},Z\right)\right)\label{eq:mapping}
\end{equation}
where $\overrightarrow{x}$ is a pattern from the original set and
$Z$ is the chromosome.
\end{enumerate}
\begin{figure}
\caption{BNF grammar of the proposed method.\label{fig:BNF-grammar-of}}

\begin{lyxcode}
S::=<expr>~~~(0)~

<expr>~::=~~(<expr>~<op>~<expr>)~~(0)~~~~~~~~~~~~~

~~~~~~~~~~~|~<func>~(~<expr>~)~~~~(1)~~~~~~~~~~~~~

~~~~~~~~~~~|<terminal>~~~~~~~~~~~~(2)~

<op>~::=~~~~~+~~~~~~(0)~~~~~~~~~~~~~

~~~~~~~~~~~|~-~~~~~~(1)~~~~~~~~~~~~~

~~~~~~~~~~~|~{*}~~~~~~(2)~~~~~~~~~~~~~

~~~~~~~~~~~|~/~~~~~~(3)

<func>~::=~~~sin~~(0)~~~~~~~~~~~~~

~~~~~~~~~~~|~cos~~(1)~~~~~~~~~~~~~

~~~~~~~~~~~|exp~~~(2)~~~~~~~~~~~~~

~~~~~~~~~~~|log~~~(3)

<terminal>::=<xlist>~~~~~~~~~~~~~~~~(0)~~~~~~~~~~~~~~~~~~~~~~

~~~~~~~~~~~|<digitlist>.<digitlist>~(1)

<xlist>::=x1~~~~(0)~~~~~~~~~~~~~~

~~~~~~~~~~~|~x2~(1)~~~~~~~~~~~~~~

~~~~~~~~~~~\dots \dots \dots{}~~~~~~~~~~~~~

~~~~~~~~~~~|~xN~(N)

<digitlist>::=<digit>~~~~~~~~~~~~~~~~~~(0)~~~~~~~~~~~~~~~~~

~~~~~~~~~~~|~<digit><digit>~~~~~~~~~~~~(1)

~~~~~~~~~~~|~<digit><digit><digit>~~~~~(2)

<digit>~~::=~0~(0)~~~~~~~~~~~~~

~~~~~~~~~~~|~1~(1)~~~~~~~~~~~~~

~~~~~~~~~~~|~2~(2)~~~~~~~~~~~~~

~~~~~~~~~~~|~3~(3)~~~~~~~~~~~~~

~~~~~~~~~~~|~4~(4)~~~~~~~~~~~~~

~~~~~~~~~~~|~5~(5)~~~~~~~~~~~~~

~~~~~~~~~~~|~6~(6)~~~~~~~~~~~~~

~~~~~~~~~~~|~7~(7)~~~~~~~~~~~~~

~~~~~~~~~~~|~8~(8)~~~~~~~~~~~~~

~~~~~~~~~~~|~9~(9)
\end{lyxcode}
\end{figure}
\begin{table}
\caption{Steps to produce a valid expression from the BNF grammar.\label{tab:table_with_steps}}

\begin{tabular}{|c|c|c|}
\hline 
String & Chromosome & Operation\tabularnewline
\hline 
\hline 
<expr> & 9,8,6,4,16,10,17,23,8,14 & $9\mod3=0$\tabularnewline
\hline 
(<expr><op><expr>) & 8,6,4,16,10,17,23,8,14 & $8\mod3=2$\tabularnewline
\hline 
(<terminal><op><expr>) & 6,4,16,10,17,23,8,14 & $6\mod2=0$\tabularnewline
\hline 
(<xlist><op><expr>) & 4,16,10,17,23,8,14 & $4\mod3=1$\tabularnewline
\hline 
(x2<op><expr>) & 16,10,17,23,8,14 & $16\mod4=0$\tabularnewline
\hline 
(x2+<expr>) & 10,17,23,8,14 & $10\mod3=1$\tabularnewline
\hline 
(x2+<func>(<expr>)) & 17,23,8,14 & $17\mod4=1$\tabularnewline
\hline 
(x2+cos(<expr>)) & 23,8,14 & $23\mod2=1$\tabularnewline
\hline 
(x2+cos(<terminal>)) & 8,14 & $8\mod2=0$\tabularnewline
\hline 
(x2+cos(<xlist>)) & 14 & $14\mod3=2$\tabularnewline
\hline 
(x2+cos(x3)) &  & \tabularnewline
\hline 
\end{tabular}
\end{table}


\subsection{Feature construction \label{subsec:Feature-construction}}

The main steps of the algorithm used in the first phase are listed
below:
\begin{enumerate}
\item \textbf{Initialization step}

\begin{enumerate}
\item \textbf{Set} iter=0, the current generation number.
\item Set $\mbox{TR}=\left\{ \left(\overrightarrow{x_{1}},y_{1}\right),\left(\overrightarrow{x_{2}},y_{2}\right),\ldots,\left(\overrightarrow{x_{M}},y_{M}\right)\right\} $
the original train set.
\item \textbf{Set} $N_{c}$ the number of chromosomes\textbf{ }and $N_{f}$
the number of constructed features.
\item \textbf{Initialize} randomly in range $[0,255]$ the integer chromosomes
$Z_{i},i=1\ldots N_{c}$ 
\item \textbf{Set} $N_{g}$ the maximum number of generations.
\item \textbf{Set} $p_{s}\in[0,1]$ as the selection rate and $p_{m}\in[0,1]$
the mutation rate.
\end{enumerate}
\item \textbf{Termination check.} \textbf{If} iter>=$N_{g}$ \textbf{goto}
step \ref{enu:Obtain-the-best}. \label{enu:Termination-check.-If}
\item \textbf{Calculate} the fitness $f_{i}$ for every chromosome $Z_{i}$
of the population:\label{enu:Calculate-the-corresponding}

\begin{enumerate}
\item \textbf{Use} the procedure described in subsection \ref{subsec:Transformation-procedure}
and create $N_{f}$ features.
\item \textbf{Create} a modified training set 
\begin{equation}
\mbox{TN}=\left\{ \left(G\left(\overrightarrow{x_{1}},Z_{i}\right),y_{1}\right),\left(G\left(\overrightarrow{x_{2}},Z_{i}\right),y_{2}\right),\ldots,\left(G\left(\overrightarrow{x_{M}},Z_{i}\right),y_{M}\right)\right\} 
\end{equation}
 
\item \textbf{Train} an RBF neural network $C$ with $H$ processing units
on the modified training set $\mbox{TN }$using the following train
error
\begin{equation}
f_{i}=\sum_{j=1}^{M}\left(C\left(G\left(\overrightarrow{x_{j}},Z_{i}\right)\right)-y_{j}\right)^{2}
\end{equation}
\end{enumerate}
\item \textbf{Genetic Operators}

\begin{enumerate}
\item \textbf{Selection procedure}: The chromosomes are sorted in descending
order according to their fitness value. The first $\left(1-p_{s}\right)\times N_{c}$
chromosomes are transferred to the next generation. The rest of the
chromosomes are substituted by offsprings created through crossover
procedure: For every offspring two chromosomes (parents) are selected
from the old population using tournament selection. The procedure
of tournament selection has as follows: A set of $K>1$ randomly selected
chromosomes is produced and the chromosome with the best fitness value
in this set is selected and the others are discarded. Each offspring
is created by the parents using the one point crossover. During one
point crossover the parent chromosomes are cut at a randomly selected
point and their right-hand side subchromosomes are exchanged.
\item \textbf{Mutation procedure:} For every element of each chromosome
a random number $r$ in range $\left[0,1\right]$ is produced. If
$r\le p_{m}$ then the corresponding element is randomly altered.
\item \textbf{Replace} the $p_{s}\times N_{c}$ worst chromosomes in the
population with the offsprings created by the genetic operators.
\end{enumerate}
\item \textbf{Set} iter=iter+1 and \textbf{goto} Step \ref{enu:Termination-check.-If}.
\item \textbf{Obtain }the best value in the population, denoted as $f_{l}$
for the corresponding chromosome \textbf{$Z_{l}$ }and Terminate.\textbf{\label{enu:Obtain-the-best}}
\end{enumerate}

\subsection{Weight decay mechanism}

The quantity $x$ in the equation \ref{eq:sig} of the sigmoid function
is calculated through many calculations involving the input patterns
as well as the weight vector. If the absolute value of $x$ is too
large i.e greater than 20, then the sigmoid function tends to be zero
or one and the neural network may lost its generalization capabilities.
In order to measure the above effect we can define the bound quantity
$B\left(N\left(\overrightarrow{x},\overrightarrow{w}\right),F\right)$
as shown in the Algorithm \ref{alg:CalculationBound}. 

\begin{algorithm}
\caption{Calculation of the bounding quantity for neural network $N(x,w)$.\label{alg:CalculationBound}}

\begin{enumerate}
\item \textbf{Set} $b=0$
\item \textbf{For} $i=1..K$ \textbf{Do}
\begin{enumerate}
\item \textbf{For} $j=1..M$ \textbf{Do}
\begin{enumerate}
\item \textbf{Set} $v=\sum_{kT=1}^{d}w_{(d+2)i-(d+i)+k}x_{jk}+w_{(d+2)i}$
\item \textbf{If} $\left|v\right|>F$ \textbf{set} $b=b+1$
\end{enumerate}
\item \textbf{EndFor}
\end{enumerate}
\item \textbf{EndFor}
\item \textbf{Return} $\frac{b}{K\star M}$
\end{enumerate}
\end{algorithm}


\subsection{Application of genetic algorithm}

The steps of the hybrid genetic algorithm used in the second phase
of the proposed algorithm are the following:
\begin{enumerate}
\item \textbf{Initialization step}

\begin{enumerate}
\item \textbf{Set} iter=0, the current generation number.
\item \textbf{Set} TN the modified training set, where
\begin{equation}
\mbox{TN}=\left\{ \left(G\left(\overrightarrow{x_{1}},Z_{l}\right),y_{1}\right),\left(G\left(\overrightarrow{x_{2}},Z_{l}\right),y_{2}\right),\ldots,\left(G\left(\overrightarrow{x_{M}},Z_{l}\right),y_{M}\right)\right\} 
\end{equation}
\item \textbf{Initialize} randomly the double precision chromosomes $D_{i},i=1\ldots N_{c}$
in range $\left[L_{N},R_{N}\right]$. The size of each chromosome
is set to $W=\left(N_{f}+2\right)H$
\end{enumerate}
\item \textbf{Termination check.} \textbf{If} iter>=$N_{g}$ \textbf{goto}
step \ref{enu:Local-Search-step.}\label{enu:Termination-check.-If-1}
\item \textbf{Fitness calculation step}.
\begin{enumerate}
\item \textbf{For} every chromosome $D_{i}$ 
\begin{enumerate}
\item \textbf{Calculate} the quantity $B_{i}=\sum_{x\in\mbox{TN}}\left(B\left(N\left(x,D_{i}\right),F\right)\right)$
using the algorithm \ref{alg:CalculationBound}.
\item \textbf{Calculate} the quantity $E_{i}=\sum_{(x,y)\in\mbox{TN}}\left(N\left(x,D_{i}\right)-y\right)^{2}$
, the training error of the neural network where the chromosome $D_{i}$
is used as the weight vector.
\item \textbf{Set} $f_{i}=-E_{i}\left(1+\lambda B_{i}^{2}\right)$, where
$\lambda>0$ as the fitness of $D_{i}$. 
\end{enumerate}
\item \textbf{End For}
\end{enumerate}
\item \textbf{Genetic operations Step}. Apply the same genetic operations
as in the first algorithm of subsection \ref{subsec:Feature-construction}.
\item \textbf{Set} iter=iter+1 and \textbf{goto} step \ref{enu:Termination-check.-If-1}
\item \textbf{Local Search step. \label{enu:Local-Search-step.}}
\begin{enumerate}
\item \textbf{Obtain }the best chromosome $D^{*}$ in the genetic population.
\item \textbf{For} i=1..W \textbf{Do}

\begin{enumerate}
\item \textbf{Set} $p_{i}=D_{i}^{*}$
\item \textbf{Set} $LM_{i}=-\alpha\left|p_{i}\right|$ 
\item \textbf{Set} $RM_{i}=\ \alpha\left|p_{i}\right|$ , $\alpha$ being
a positive number with $\alpha>1$ .
\end{enumerate}
\item \textbf{EndFor}
\item \textbf{Set} $L^{*}=\mathcal{L}\left(D^{*},LM,RM\right)$ where $\mathcal{L}()$
is a local optimization method that searches for a local optimum of
$N\left(x,D^{*}\right)$ inside the bounds $\left[\overrightarrow{LM},\overrightarrow{RM}\right]$.
The TOLMIN \cite{tolmin} local optimization procedure used in the
above algorithm, which is modified version BFGS (Broyden--Fletcher--Goldfarb--Shanno)
local optimization procedure\cite{bfgs2}.
\item \textbf{Apply} the optimized neural network $N\left(x,D^{*}\right)$
to the test set, that has been modified using the same transformation
procedure as in the train set and report the final results.
\end{enumerate}
\end{enumerate}

\section{Experiments \label{sec:Experiments}}

The proposed method was tested against neural network trained by a
genetic algorithm (denoted as MLP GEN) on a series of classification
and regression datasets. The software for the algorithm was coded
using ANSI C++ and all the experiments were conducted using the OpenMP
library \cite{openmp} for parallelization. The experiments were executed
30 times using different seed for the random generator each time and
averages were taken. For classification datasets the average classification
error on the test set is reported and for regression datasets the
average mean squared error on the test set is shown.\textbf{ }10 fold
cross validation is used in the conducted experiments. The parameters
used in the experiments are listed in Table \ref{tab:Experimental-parameters.}.

\begin{table}
\caption{Experimental parameters.\label{tab:Experimental-parameters.}}

\centering{}%
\begin{tabular}{|c|c|}
\hline 
PARAMETER & VALUE\tabularnewline
\hline 
\hline 
$H$ & 10\tabularnewline
\hline 
$N_{c}$ & 500\tabularnewline
\hline 
$N_{f}$ & 2\tabularnewline
\hline 
$p_{s}$ & 0.10\tabularnewline
\hline 
$p_{m}$ & 0.05\tabularnewline
\hline 
$N_{g}$ & 200\tabularnewline
\hline 
$L_{N}$ & -10.0\tabularnewline
\hline 
$R_{N}$ & 10.0\tabularnewline
\hline 
$F$ & 20.0\tabularnewline
\hline 
$\lambda$ & 100.0\tabularnewline
\hline 
$\alpha$ & 5.0\tabularnewline
\hline 
\end{tabular}
\end{table}


\subsection{Experimental datasets }

The following classification datasets were acquired from the Machine
Learning Repository \url{http://www.ics.uci.edu/~mlearn/MLRepository.html  }
and from \url{https://sci2s.ugr.es/keel/}: 
\begin{enumerate}
\item \textbf{Balance} dataset: Used to model psychological experimental
results. The dataset has 625 instances with 4 features each.
\item \textbf{Dermatology} dataset: Dataset used for differential diagnosis
of erythemato-squamous diseases. The dataset has 366 instances of
34 features each. 
\item \textbf{Glass} dataset. This dataset contains glass component analysis
for glass pieces that belong to 6 classes. The dataset contains 214
examples with 10 features each. 
\item \textbf{Hayes Roth} dataset: This dataset\cite{hayesroth} contains
\textbf{5} numeric-valued attributes and 132 patterns. 
\item \textbf{Heart} dataset: Dataset used to discriminate between absence
or presence of heart disease. The dataset has 270 instances of 13
features each.
\item \textbf{Ionosphere} dataset: The ionosphere dataset (ION in the following
tables) contains data from the Johns Hopkins Ionosphere database.
The two-class dataset contains 351 examples of 34 features each. 
\item \textbf{Parkinsons} dataset: This dataset\cite{parkinson} is composed
of a range of biomedical voice measurements from 31 people, 23 with
Parkinson's disease (PD).The dataset has 22 features. 
\item \textbf{Pima }dataset: The Pima Indians Diabetes dataset contains
768 examples of 8 attributes each that are classified into two categories:
healthy and diabetic.
\item \textbf{PopFailures} dataset: Dataset used in meteorology. The dataset
has 540 instances of 18 features each.
\item \textbf{Spiral} dataset: The spiral artificial dataset contains 1000
two-dimensional examples that belong to two classes (500 examples
each). The number of the features is 2. The data in the first class
are created using the following formula: $x_{1}=0.5t\cos\left(0.08t\right),\ x_{2}=0.5t\cos\left(0.08t+\frac{\pi}{2}\right)$
and the second class data using\textbf{: $x_{1}=0.5t\cos\left(0.08t+\pi\right),\ x_{2}=0.5t\cos\left(0.08t+\frac{3\pi}{2}\right)$}
\item \textbf{Wine}: The wine recognition dataset (WINE) contains data from
wine chemical analysis. It contains 178 examples of 13 features each
that are classified into three classes. 
\item \textbf{Wdbc} dataset: The Wisconsin diagnostic breast cancer dataset
(WDBC) contains data for breast tumors. The dataset has 30 features. 
\end{enumerate}
Also an EEG dataset was used, that is is available from \cite{eeg1,eeg2}
and includes recordings for both healthy and epileptic subjects, is
used. The dataset includes five subsets (denoted as Z, O, N, F, and
S) each containing 100 single-channel EEG segments, each one having
23.6-second duration. In the current work two variants of the dataset
was used: 

\begin{enumerate}
\item In the first (\textbf{ZO\_NF\_S}), all the EEG segments from the dataset
were used and they were classified into three different classes: Z
and O types of EEG segments were combined to a single class, N and
F types were also combined to a single class, and type S was the third
class. This set is the one closest to real medical applications including
three categories; normal (i.e., types Z and O), seizure free (i.e.,
types N and F) and seizure (i.e., type S). 
\item In the second (\textbf{Z\_F\_S}) dataset, which is similar with the
first one, a subset of the EEG segments from the dataset were employed.
The normal class includes only the Z-type EEG segments, the seizure-free
class the F-type EEG segments, and the seizure class the S-type. 
\item In the third (\textbf{Z\_O\_N\_F\_S}) dataset The dataset consists
of five sets (denoted as Z, O, N, F and S) each containing 100 single-channel
EEG segments each having 23.6 sec duration. Sets Z and O have been
taken from surface EEG recordings of five healthy volunteers with
eye open and closed, respectively. Signals in two sets have been measured
in seizure-free intervals from five patients in the epileptogenic
zone (F) and from the hippocampal formation of the opposite hemisphere
of the brain (N). Set S contains seizure activity, selected from all
recording sites exhibiting ictal activity. Sets Z and O have been
recorded extracranially, whereas sets N, F and S have been recorded
intracranially.
\end{enumerate}
The regression datasets are available from the Statlib URL \url{ftp://lib.stat.cmu.edu/datasets/index.html }
and other sources: 
\begin{enumerate}
\item \textbf{BK} dataset. This dataset comes from Smoothing Methods in
Statistics \cite{Stat} and is used to estimate the points scored
per minute in a basketball game. The dataset has 96 patterns of 4
features each. 
\item \textbf{BL} dataset: This dataset can be downloaded from StatLib.
It contains data from an experiment on the affects of machine adjustments
on the time to count bolts. It contains 40 patters of 7 features each.
\item \textbf{Housing} dataset. This dataset was taken from the StatLib
library which is maintained at Carnegie Mellon University and it is
described in \cite{key23}.
\item \textbf{Laser} dataset. Dataset used in laser experiments. The dataset
has 993 instances of 4 features each.
\item \textbf{NT} dataset. This dataset contains data from \cite{ntdataset}
that examined whether the true mean body temperature is 98.6 F. The
number of patterns is 131 and the number of features 2. 
\item \textbf{Quake} dataset. The objective here is to approximate the strength
of a earthquake. The dataset has 2178 instances of 3 features each. 
\item \textbf{FA} dataset. The FA dataset contains percentage of body fat,
age, weight, height, and ten body circumference measurements. The
goal is to fit body fat to the other measurements. The number of the
features is 18. The total number of patterns is 252. 
\item \textbf{PY} dataset (Pyrimidines problem). The source of this dataset
is the URL: \url{https://www.dcc.fc.up.pt/~ltorgo/Regression/DataSets.html}
and it is a problem of 27 attributes and 74 number of patterns. The
task consists of Learning Quantitative Structure Activity Relationships
(QSARs) and provided by \cite{pydataset}. 
\end{enumerate}

\subsection{Experimental results}

For the case of classification datasets the results are reported in
Table \ref{tab:exp1} and for the case of regression datasets the
results are reported in Table \ref{tab:exp2}. The column MLP GEN
denotes the neural network with $H$ hidden nodes that was trained
by a genetic algorithm with $N_{g}$ chromosomes and the column FC
MLP denotes the proposed method. Also, the column GAIN indicates the
percentage gain in test error (classification or regression) obtained
by the proposed method. From the conducted experiments it is evident
that the proposed method outperforms the simple genetic algorithm
in terms of efficiency. Also, the percentage gain is too high in many
cases like in the regression datasets. On the other hand the proposed
method is slower than a simple genetic algorithm because it relies
on two steps and every step involves the usage of a genetic algorithm. 

\begin{table}
\caption{Experimental results for classification datasets.\label{tab:exp1}}

\centering{}%
\begin{tabular}{|c|c|c|c|}
\hline 
DATASET & MLP GEN & FC MLP & \textbf{GAIN}\tabularnewline
\hline 
\hline 
BALANCE & 8.23\% & 0.30\% & \textbf{96.35\%}\tabularnewline
\hline 
DERMATOLOGY & 10.01\% & 4.98\% & \textbf{50.25\%}\tabularnewline
\hline 
GLASS & 58.03\% & 45.84\% & \textbf{21.00\%}\tabularnewline
\hline 
HAYES ROTH & 35.26\% & 23.26\% & \textbf{34.03\%}\tabularnewline
\hline 
HEART & 25.46\% & 17.71\% & \textbf{30.44\%}\tabularnewline
\hline 
IONOSPHERE & 13.67\% & 8.42\% & \textbf{38.41\%}\tabularnewline
\hline 
PARKINSONS & 17.47\% & 10.10\% & \textbf{36.46\%}\tabularnewline
\hline 
PIMA & 32.98\% & 23.76\% & \textbf{27.96\%}\tabularnewline
\hline 
POPFAILURES & 7.66\% & 4.66\% & \textbf{39.16\%}\tabularnewline
\hline 
SPIRAL & 45.71\% & 26.53\% & \textbf{41.96\%}\tabularnewline
\hline 
WINE & 20.82\% & 7.31\% & \textbf{64.89\%}\tabularnewline
\hline 
WDBC & 6.32\% & 3.47\% & \textbf{45.09\%}\tabularnewline
\hline 
Z\_F\_S & 9.42\% & 5.52\% & \textbf{41.40\%}\tabularnewline
\hline 
Z\_O\_N\_F\_S & 60.38\% & 31.20\% & \textbf{48.33\%}\tabularnewline
\hline 
ZO\_NF\_S & 8.06\% & 4.00\% & \textbf{50.37\%}\tabularnewline
\hline 
\end{tabular}
\end{table}
\begin{table}
\caption{Experiments for regression datasets.\label{tab:exp2}}

\centering{}%
\begin{tabular}{|c|c|c|c|}
\hline 
DATASET & MLP GEN & FC MLP & \textbf{GAIN}\tabularnewline
\hline 
\hline 
BK & 0.21 & 0.03 & \textbf{85.71\%}\tabularnewline
\hline 
BL & 0.84 & 0.005 & \textbf{99.40\%}\tabularnewline
\hline 
Housing & 30.05 & 10.77 & \textbf{64.16\%}\tabularnewline
\hline 
Laser & 0.003 & 0.002 & \textbf{33.33\%}\tabularnewline
\hline 
NT & 1.11 & 0.01 & \textbf{99.10\%}\tabularnewline
\hline 
Quake & 0.07 & 0.03 & \textbf{57.14\%}\tabularnewline
\hline 
FA & 0.04 & 0.01 & \textbf{75.00\%}\tabularnewline
\hline 
PY & 0.21 & 0.02 & \textbf{90.47\%}\tabularnewline
\hline 
\end{tabular}
\end{table}


\section{Conclusions\label{sec:Conclusions}}

A hybrid method was proposed here for classification and regression
problems. The method is composed by two steps: during the first step
a recently introduced method was used to create artificial features
from the original ones. The feature construction method has utilized
the commonly used technique of Grammatical Evolution to produce artificial
features. The outcome of the first step was the modified train and
test sets of the objective problem. During the second step, a hybrid
genetic algorithm which preserves the generalization abilities of
the neural network was incorporated. The proposed method was tested
on a variety of classification and regression problems and the results
were very promising.

\subsection*{Compliance with Ethical Standards }

All authors declare that they have no has no conflict of interest. 
\begin{thebibliography}{10}
\bibitem{nn1}C. Bishop, Neural Networks for Pattern Recognition,
Oxford University Press, 1995.

\bibitem{nn2}G. Cybenko, Approximation by superpositions of a sigmoidal
function, Mathematics of Control Signals and Systems \textbf{2}, pp.
303-314, 1989.

\bibitem{nnphysics1}P. Baldi, K. Cranmer, T. Faucett et al, Parameterized
neural networks for high-energy physics, Eur. Phys. J. C \textbf{76},
2016.

\bibitem{nnphysics2}J. J. Valdas and G. Bonham-Carter, Time dependent
neural network models for detecting changes of state in complex processes:
Applications in earth sciences and astronomy, Neural Networks \textbf{19},
pp. 196-207, 2006

\bibitem{nnphysics3}G. Carleo,M. Troyer, Solving the quantum many-body
problem with artificial neural networks, Science \textbf{355}, pp.
602-606, 2017.

\bibitem{nnchem1}Lin Shen, Jingheng Wu, and Weitao Yang, Multiscale
Quantum Mechanics/Molecular Mechanics Simulations with Neural Networks,
Journal of Chemical Theory and Computation \textbf{12}, pp. 4934-4946,
2016.

\bibitem{nnchem2}Sergei Manzhos, Richard Dawes, Tucker Carrington,
Neural network-based approaches for building high dimensional and
quantum dynamics-friendly potential energy surfaces, Int. J. Quantum
Chem. \textbf{115}, pp. 1012-1020, 2015.

\bibitem{nnchem3}Jennifer N. Wei, David Duvenaud, and Al�n Aspuru-Guzik,
Neural Networks for the Prediction of Organic Chemistry Reactions,
ACS Central Science \textbf{2}, pp. 725-732, 2016.

\bibitem{nnecon1}Lukas Falat and Lucia Pancikova, Quantitative Modelling
in Economics with Advanced Artificial Neural Networks, Procedia Economics
and Finance \textbf{34}, pp. 194-201, 2015.

\bibitem{nnecon2}Mohammad Namazi, Ahmad Shokrolahi, Mohammad Sadeghzadeh
Maharluie, Detecting and ranking cash flow risk factors via artificial
neural networks technique, Journal of Business Research \textbf{69},
pp. 1801-1806, 2016.

\bibitem{nncecon3}G. Tkacz, Neural network forecasting of Canadian
GDP growth, International Journal of Forecasting \textbf{17}, pp.
57-69, 2001.

\bibitem{nnmed1}Igor I. Baskin, David Winkler and Igor V. Tetko,
A renaissance of neural networks in drug discovery, Expert Opinion
on Drug Discovery \textbf{11}, pp. 785-795, 2016.

\bibitem{nnmed2}Ronadl Bartzatt, Prediction of Novel Anti-Ebola Virus
Compounds Utilizing Artificial Neural Network (ANN), Chemistry Faculty
Publications \textbf{49}, pp. 16-34, 2018.

\bibitem{nnc}I.G. Tsoulos, D. Gavrilis, E. Glavas, Neural network
construction and training using grammatical evolution, Neurocomputing
\textbf{72}, pp. 269-277, 2008.

\bibitem{bpnn}D.E. Rumelhart, G.E. Hinton and R.J. Williams, Learning
representations by back-propagating errors, Nature \textbf{323}, pp.
533 - 536 , 1986.

\bibitem{bpnn2}T. Chen and S. Zhong, Privacy-Preserving Backpropagation
Neural Network Learning, IEEE Transactions on Neural Networks \textbf{20},
, pp. 1554-1564, 2009.

\bibitem{rpropnn}M. Riedmiller and H. Braun, A Direct Adaptive Method
for Faster Backpropagation Learning: The RPROP algorithm, Proc. of
the IEEE Intl. Conf. on Neural Networks, San Francisco, CA, pp. 586--591,
1993.

\bibitem{rpropnn3}T. Pajchrowski, K. Zawirski and K. Nowopolski,
Neural Speed Controller Trained Online by Means of Modified RPROP
Algorithm, IEEE Transactions on Industrial Informatics \textbf{11},
pp. 560-568, 2015.

\bibitem{rpropnn2}Rinda Parama Satya Hermanto, Suharjito, Diana,
Ariadi Nugroho, Waiting-Time Estimation in Bank Customer Queues using
RPROP Neural Networks, Procedia Computer Science \textbf{ 135}, pp.
35-42, 2018.

\bibitem{quasinn}B. Robitaille and B. Marcos and M. Veillette and
G. Payre, Modified quasi-Newton methods for training neural networks,
Computers \& Chemical Engineering \textbf{20}, pp. 1133-1140, 1996.

\bibitem{quasinn2}Q. Liu, J. Liu, R. Sang, J. Li, T. Zhang and Q.
Zhang, Fast Neural Network Training on FPGA Using Quasi-Newton Optimization
Method,IEEE Transactions on Very Large Scale Integration (VLSI) Systems
\textbf{26}, pp. 1575-1579, 2018.

\bibitem{geneticnn}F. H. F. Leung, H. K. Lam, S. H. Ling and P. K.
S. Tam, Tuning of the structure and parameters of a neural network
using an improved genetic algorithm, IEEE Transactions on Neural Networks
\textbf{14}, pp. 79-88, 2003

\bibitem{geneticnn2}X. Yao, Evolving artificial neural networks,
Proceedings of the IEEE, 87(9), pp. 1423-1447, 1999.

\bibitem{psonn}C. Zhang, H. Shao and Y. Li, Particle swarm optimisation
for evolving artificial neural network, IEEE International Conference
on Systems, Man, and Cybernetics, , pp. 2487-2490, 2000.

\bibitem{psonn2}Jianbo Yu, Shijin Wang, Lifeng Xi, Evolving artificial
neural networks using an improved PSO and DPSO \textbf{71}, pp. 1054-1060,
2008.

\bibitem{nndimension}Verleysen M., Francois D., Simon G., Wertz V.,
On the effects of dimensionality on data analysis with neural networks.
In: Mira J., �lvarez J.R. (eds) Artificial Neural Nets Problem Solving
Methods. IWANN 2003. Lecture Notes in Computer Science, vol 2687.
Springer, Berlin, Heidelberg. 2003.

\bibitem{nnpca1}Burcu Erkmen, T�lay Y\i ld\i r\i m, Improving classification
performance of sonar targets by applying general regression neural
network with PCA, Expert Systems with Applications \textbf{35}, pp.
472-475, 2008.

\bibitem{nnpca2}Jing Zhou, Aihuang Guo, Branko Celler, Steven Su,
Fault detection and identification spanning multiple processes by
integrating PCA with neural network, Applied Soft Computing \textbf{14},
pp. 4-11, 2014.

\bibitem{nnpca3}Ravi Kumar G., Nagamani K., Anjan Babu G., A Framework
of Dimensionality Reduction Utilizing PCA for Neural Network Prediction.
In: Borah S., Emilia Balas V., Polkowski Z. (eds) Advances in Data
Science and Management. Lecture Notes on Data Engineering and Communications
Technologies, vol 37. Springer, Singapore. 2020

\bibitem{nngeman}S. Geman, E. Bienenstock and R. Doursat, Neural
networks and the bias/variance dilemma, Neural Computation 4 , pp.
1 - 58, 1992.

\bibitem{nnhawkins}Douglas M. Hawkins, The Problem of Overfitting,
J. Chem. Inf. Comput. Sci. \textbf{44}, pp. 1--12, 2004.

\bibitem{nnsharing1}S.J. Nowlan and G.E. Hinton, Simplifying neural
networks by soft weight sharing, Neural Computation 4, pp. 473-493,
1992.

\bibitem{nnprunning1}S.J. Hanson and L.Y. Pratt, Comparing biases
for minimal network construction with back propagation, In D.S. Touretzky
(Ed.), Advances in Neural Information Processing Systems, Volume 1,
pp. 177-185, San Mateo, CA: Morgan Kaufmann, 1989.

\bibitem{nnprunning2}M.C. Mozer and P. Smolensky, Skeletonization:
a technique for trimming the fat from a network via relevance assesment.
In D.S. Touretzky (Ed.), Advances in Neural Processing Systems, Volume
1, pp. 107-115, San Mateo CA: Morgan Kaufmann, 1989.

\bibitem{nnprunning3}M. Augasta and T. Kathirvalavakumar, Pruning
algorithms of neural networks --- a comparative study, Central European
Journal of Computer Science, 2003.

\bibitem{nndrop1}Nitish Srivastava, G E Hinton, Alex Krizhevsky,
Ilya Sutskever, Ruslan R Salakhutdinov, Dropout: a simple way to prevent
neural networks from overfitting, Journal of Machine Learning Research
\textbf{15}, pp. 1929-1958, 2014.

\bibitem{nnearly1}Lutz Prechelt, Automatic early stopping using cross
validation: quantifying the criteria, Neural Networks \textbf{11},
pp. 761-767, 1998.

\bibitem{nnearly2}X. Wu and J. Liu, A New Early Stopping Algorithm
for Improving Neural Network Generalization, 2009 Second International
Conference on Intelligent Computation Technology and Automation, Changsha,
Hunan, 2009, pp. 15-18.

\bibitem{nndecay1}N. K. Treadgold and T. D. Gedeon, Simulated annealing
and weight decay in adaptive learning: the SARPROP algorithm,IEEE
Transactions on Neural Networks \textbf{9}, pp. 662-668, 1998.

\bibitem{nndecay2}M. Carvalho and T. B. Ludermir, Particle Swarm
Optimization of Feed-Forward Neural Networks with Weight Decay, 2006
Sixth International Conference on Hybrid Intelligent Systems (HIS'06),
Rio de Janeiro, Brazil, 2006, pp. 5-5.

\bibitem{ge1}M. O\textquoteright Neill, C. Ryan, Grammatical evolution,
IEEE Trans. Evol. Comput. \textbf{5,}pp. 349--358, 2001.

\bibitem{fc1}Dimitris Gavrilis, Ioannis G. Tsoulos, Evangelos Dermatas,
Selecting and constructing features using grammatical evolution, Pattern
Recognition Letters \textbf{29},pp. 1358-1365, 2008. 

\bibitem{fc2}Dimitris Gavrilis, Ioannis G. Tsoulos, Evangelos Dermatas,
Neural Recognition and Genetic Features Selection for Robust Detection
of E-Mail Spam, Advances in Artificial Intelligence Volume 3955 of
the series Lecture Notes in Computer Science pp 498-501, 2006.

\bibitem{fc3}George Georgoulas, Dimitris Gavrilis, Ioannis G. Tsoulos,
Chrysostomos Stylios, Jo�o Bernardes, Peter P. Groumpos, Novel approach
for fetal heart rate classification introducing grammatical evolution,
Biomedical Signal Processing and Control \textbf{2},pp. 69-79, 2007 

\bibitem{fc4}Otis Smart, Ioannis G. Tsoulos, Dimitris Gavrilis, George
Georgoulas, Grammatical evolution for features of epileptic oscillations
in clinical intracranial electroencephalograms, Expert Systems with
Applications \textbf{38}, pp. 9991-9999, 2011

\bibitem{rbf1}J. Park and I. W. Sandberg, Universal Approximation
Using Radial-Basis-Function Networks, Neural Computation 3, pp. 246-257,
1991.

\bibitem{tolmin}M.J.D. Powell, A Tolerant Algorithm for Linearly
Constrained Optimization Calculations, Mathematical Programming \textbf{45},
pp 547, 1989.

\bibitem{bfgs2}R. Fletcher, A new approach to variable metric algorithms,
Computer Journal \textbf{13}, pp. 317-322, 1970. 

\bibitem{openmp}R. Chandra, L. Dagum, D. Kohr, D. Maydan,J. McDonald
and R. Menon, Parallel Programming in OpenMP, Morgan Kaufmann Publishers
Inc., 2001.

\bibitem{hayesroth}B. Hayes-Roth, B., F. Hayes-Roth. Concept learning
and the recognition and classification of exemplars. Journal of Verbal
Learning and Verbal Behavior \textbf{16}, pp. 321-338, 1977.

\bibitem{parkinson}Max A. Little, Patrick E. McSharry, Eric J. Hunter,
Lorraine O. Ramig (2008), 'Suitability of dysphonia measurements for
telemonitoring of Parkinson's disease', IEEE Transactions on Biomedical
Engineering \textbf{56}, pp. 1015-1022, 2009.

\bibitem{eeg1}R. G. Andrzejak, K. Lehnertz, F.Mormann, C. Rieke,
P. David, and C. E. Elger, \textquotedblleft Indications of nonlinear
deterministic and finite-dimensional structures in time series of
brain electrical activity: dependence on recording region and brain
state,\textquotedblright{} Physical Review E, vol. 64, no. 6, Article
ID 061907, 8 pages, 2001. 

\bibitem{eeg2}A. T. Tzallas, M. G. Tsipouras, and D. I. Fotiadis,
\textquotedblleft Automatic Seizure Detection Based on Time-Frequency
Analysis and Artificial Neural Networks,\textquotedblright{} Computational
Intelligence and Neuroscience, vol. 2007, Article ID 80510, 13 pages,
2007. doi:10.1155/2007/80510

\bibitem{Stat}J.S. Simonoff, Smooting Methods in Statistics, Springer
- Verlag, 1996.

\bibitem{key23}D. Harrison and D.L. Rubinfeld, Hedonic prices and
the demand for clean ai, J. Environ. Economics \& Management \textbf{5},
pp. 81-102, 1978.

\bibitem{ntdataset}Mackowiak, P.A., Wasserman, S.S., Levine, M.M.,
1992. A critical appraisal of 98.6 degrees f, the upper limit of the
normal body temperature, and other legacies of Carl Reinhold August
Wunderlich. J. Amer. Med. Assoc. 268, 1578--1580

\bibitem{pydataset}R.D. King, S. Muggleton, R. Lewis, M.J.E. Sternberg,
Proc. Nat. Acad. Sci. USA \textbf{89}, pp. 11322--11326, 1992. 
\end{thebibliography}

\end{document}
